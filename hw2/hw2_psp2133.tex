\documentclass[fleqn]{article}
\usepackage[T1]{fontenc}
\usepackage[utf8]{inputenc}
\usepackage{amsmath}
\usepackage{dsfont}
\usepackage{mathtools}
\usepackage{algorithm}
\usepackage{fontspec}
\usepackage{url}

\title{Machine Learning - Homework 2}
\author{Parita Pooj (psp2133)}
\date{September 28, 2016}

\newcommand\tab[1][0.6cm]{\hspace*{#1}}

\begin{document}
\maketitle
\setcounter{secnumdepth}{0}
\section{Problem 1}{}
\begin{itemize}
        \item[(a)]
                $\mu_{y, j} = \smashoperator{\sum_{i = 1}^{n}} \frac{\mathds{1}\{y_i = y\} x_{i, j}}{\mathds{1}\{y_i = y\}}$
        \item[(b)]
                Source code: hw2\_p1b.m\\
                Training Error Rate: 21.6257\%\\
                Test Error Rate: 37.6016\%\\
                Running Time: ~30 secs
        \item[(c)]
                Source Code: hw2\_p1c.m\\
                Training Error Rate: 5.7794\%\\
                Test Error Rate: 13.1383\%\\
                Running Time: ~1.15 secs
        \item[(d)]
                Source Code: hw2\_p1d.m\\
                $\alpha_0$ = -20.7849\\
                \textbf{20 most positive words (corresponding to 20 largest values of $\alpha_j$)}\\
                1: 'firearms'\\
                2: 'occupied'\\
                3: 'israelis'\\
                4: 'serdar'\\
                5: 'argic'\\
                6: 'ohanus'\\
                7: 'appressian'\\
                8: 'sahak'\\
                9: 'melkonian'\\
                10: 'villages'\\
                11: 'cramer'\\
                12: 'armenia'\\
                13: 'cpr'\\
                14: 'sdpa'\\
                15: 'handgun'\\
                16: 'optilink'\\
                17:'palestine'\\
                18: 'firearm'\\
                19: 'budget'\\
                20: 'arabs'\\

                \textbf{20 most negative words (corresponding to 20 smallest values of $\alpha_j$)}\\
                1: 'athos'\\
                2: 'atheism'\\
                3: 'atheists'\\
                4: 'clh'\\
                5: 'teachings'\\
                6: 'revelation'\\
                7: 'testament'\\
                8: 'livesey'\\
                9: 'atheist'\\
                10: 'wpd'\\
                11: 'solntze'\\
                12: 'scriptures'\\
                13: 'theology'\\
                14: 'believers'\\
                15: 'ksand'\\
                16: 'alink'\\
                17: 'benedikt'\\
                18: 'jesus'\\
                19: 'prophet'\\
                20: 'mozumder'\\
\end{itemize}
\section{Problem 2}{}
\begin{itemize}
	\item[(a)] The classifier $f^*$ predicts 1 when $c \frac{2}{3}N(0,1) \leq \frac{1}{3}N(2, \frac{1}{4})$\\
		Thus, the range in which $f^*$ predicts 1 can be given by the roots of:\\
		$c \frac{2}{3}N(0,1) = \frac{1}{3}N(2, \frac{1}{4})$\\
		Thus, we get the quadratic equation\\
		$3x^2 - 16x + 16 + 2log(c) = 0$\\
		which give the roots:\\
		\[ x = \frac{16 \pm \sqrt{256 - 12(16 + 2log(c))}}{6} \]
		\[ x = \frac{16 \pm \sqrt{64 - 24log(c)}}{6} \]
		For $1 \leq c \leq 14$,\\
		$f^*$ predicts 1 when $x$ is in the interval $[\frac{16 - \sqrt{64 - 24log(c)}}{6}, \frac{16 + \sqrt{64 - 24log(c)}}{6}]$\\

        \item[(b)] 
		For $c \geq 15$, the roots are imaginary i.e. they don't exist. Thus, the classifier will always predict 0.\\
\end{itemize}
\section{Problem 3}{}
\begin{itemize}
        \item[(a)]
                Covariance matrix can be written as:\\
                $\Sigma = U \Lambda U^T$ where $U$ is orthonormal\\
                Thus, $\Sigma + \sigma^2 I = U \Lambda U^T + \sigma^2 U U^T$\\
                $\Sigma + \sigma^2 I = U (\Lambda + \sigma^2 I) U^T$\\
                Thus, the eigenvalues of $\Sigma + \sigma^2I$ are \\
                \[
                \lambda_1 + \sigma^2, \lambda_2 + \sigma^2, \dots, 
                \lambda_d + \sigma^2
                \]
        \item[(b)]
                Similarly, we get the eigenvalues of $(\Sigma + \sigma^2I)^{-2}$
                as \\
                \[
                        (\lambda_1 + \sigma^2)^{-2}, (\lambda_2 + \sigma^2)^{-2}
                        ,\dots, (\lambda_d + \sigma^2)^{-2}
                \]\\
\end{itemize}

\begin{thebibliography}{99}
        \bibitem{[1]} \url{https://www.cs.ubc.ca/~murphyk/Teaching/Stat406-Spring08/Lectures/linalg1.pdf}
        \bibitem{[2]} \url{http://cs229.stanford.edu/section/gaussians.pdf}
        \bibitem{[3]} \url{https://ocw.mit.edu/courses/mathematics/18-06sc-linear-algebra-fall-2011/least-squares-determinants-and-eigenvalues/diagonalization-and-powers-of-a/MIT18_06SCF11_Ses2.9sum.pdf}

\end{thebibliography}
\end{document}
